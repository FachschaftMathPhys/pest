% Draws horizontal bars for the given items. 100%, or a filled bar, are
% given in #3 argument. Expects a comma separated list of descriptions
% in #1 and the absolute occurrences for that items in #2. The last item
% given will be printed on top. Example usage:
% \singleHorizontalBars{first item, second item}{20,150}{150}
\newcommand{\multiHorizontalBars}[3]{
  % calculate amount of arguments as well as ymax and height. These
  % cannot directly be calculated in the axis environment, therefore it
  % is done here.
  \StrCount{#2,}{,}[\argumentCount]
  \pgfmathparse{\argumentCount+1}
  \let\calcymax\pgfmathresult
  % the "2" is fixed space that is used regardless of how many answers
  % there are. The 0.5 is the amount of space a single answer takes up.
  \pgfmathparse{\argumentCount*0.5+2}
  \let\calcheight\pgfmathresult

  % generate the coordinates which will later be parsed. The y-value
  % is simply counted up while the x-value is taken from #1. Also
  % generates the extra y ticks string.
  \gdef\coordAString{}
  \gdef\extraYTicksString{}
  \foreach \y in {1,...,\argumentCount}
    \findNthArgument{#2}{\numexpr\y-1}
    \xdef\coordAString{\coordAString (\lastFindNthArgument, \y)}
    \xdef\extraYTicksString{\extraYTicksString,\y}
  ;
  % remove superfluous comma at the start
  \StrBehind{\extraYTicksString}{,}[\extraYTicksString]
  % calculate the "not checked" side of the plot to simulate an x%
  % filled bar, which is not possible out of the box with pgfplots.
  \gdef\coordBString{}
  \foreach \y in {1,...,\argumentCount}
    \findNthArgument{#2}{\numexpr\y-1}
    \pgfmathparse{#3-\lastFindNthArgument}
    \xdef\coordBString{\coordBString (\pgfmathresult, \y)}
  ;

  % move much further to the right. The right side of the question is
  % used as starting point, and the diagram will be placed at the
  % coordinates below with the left end of the top bar. In other words:
  % the description of each bar is not taken into account and might
  % overflow into the question text.
  \path (question text.north east) ++(4cm,0) coordinate (question text end);

  \begin{axis}[
    name=horizontalBars,
    at={(question text end)},
    anchor=north west,
    % make it xbar so that we get horizontal bars and stacked, so that
    % we can easily draw a partly filled bar.
    xbar stacked,
    % set the limits so each bar appears nicely
    xmin=0, xmax=#3, ymin = 0, ymax = \calcymax,
    height = {\calcheight cm},
    bar width = 0.4cm,
    width=7cm,
    % hide the box around the plot
    axis x line=none,
    every outer y axis line/.append style={white},
    % hide x labels
    xtick=\empty,
    % hide little grey markers on top or below the actual plot
    major y tick style={draw=none},
    minor y tick style={draw=none},
    % define where to place ticks. Normal ticks are placed to the left
    % and contain each item’s description. Extra ticks contain the
    % absolute occurence of that item and are placed to the right.
    ytick=data,
    yticklabels={#1},
    extra y ticks={\extraYTicksString},
    extra y tick labels={#2},
    extra y tick style={ytick pos=right, yticklabel pos=right},
    % make them a little bit smaller
    tick label style={font=\footnotesize},
    ]
    % draw the filled part. Since \addplot doesn’t expand variables, we
    % need to do it manually so that \coordAString may be read correctly.
    \edef\temp{\noexpand\addplot[black,fill=black] coordinates {\coordAString};}
    \temp
    % draw the empty part
    \edef\temp{\noexpand\addplot[black,fill=white] coordinates {\coordBString};}
    \temp
  \end{axis}
}
